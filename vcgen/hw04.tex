\documentclass[11pt]{article}
\usepackage{homework}
\usepackage{enumerate}
\usepackage{courier}
\usepackage{url}
\usepackage{textcomp}
\sethwtitle{Software Analysis and Verification - Fall 2014}
\sethwauthor{Ruzica Piskac}%
\sethwnumber{01}%
\sethwdate{\today}
\makehwheader%

\begin{document}
\makehwtitle%


Please submit your solution to your Dropbox at the classes*v2 system.\\

The deadline for Homework 04 is October 07, 1:00PM.

\section{Verification Condition Generation}

Build a verification condition generator (VCG) based on computation of weakest
preconditions, as described in the lectures.

Your VCG should compute verification conditions, based on the code and
provided annotations. To prove the resulting formulas, connect your VCG to a
theorem prover to prove the verification conditions. You should use SMT-LIB
format and an SMT solver of your choice.

%%%%%%%%%%%%%%%%%%%%
A code skeleton is provided on Yale's private Github service.  You must be
connected to Yale network to be able to use it (simply use Yale's wifi).

\begin{enumerate}
\item Install Scala on your computer.
\item Clone the skeleton repository (\texttt{git clone https://git.yale.edu/qc35/cs454.git}).
\item Go to the directory \texttt{cs454/vcgen}.
\item Run \texttt{make}, read the output and test the parser on an example.
\item Edit \texttt{src/vcgen.scala}, fix bugs, rinse and repeat.
\item To submit, compress the \texttt{vcgen} and use the Classes*v2 dropbox.
\end{enumerate}

Start the step~5 by modifying the parser to handle logical assertions,
pre- and postconditions and loop invariants (inspire your code from what
is already in the skeleton).  After that, write a recursive function that
goes through the program AST and compute the weakest precondition.  Only
at the end, write a printing function that outputs the verification condition
using the SMT-LIB format.

It is possible that we issue some amendments to the language or that I
add extra material.  If so, I will signal it in a Classes*v2 announcement
and you will use \texttt{git pull} to fetch the changes.
%%%%%%%%%%%%%%%%%%%%

You should submit your five (or more) benchmarks until October 06, 5PM (strict
deadline!) so that we can prepare them for the competition.  Your benchmarks
should include at least five programs in the original IMP language with loops.

%%%%%%%%%%%%%%%%%%%%
To earn bonus points, extend the IMP programming language with extra features
and adapt your VCG program to handle them.  Provide a quick justification of
the code you use in your VCG to generate weakest preconditions of these extensions.

\begin{iterate}
\item Pascal for loop (\texttt{for i = n to m do ... end}).
\item Parallel assignment (\texttt{x,y := y,x;}.
\item Arrays.
\end{iterate}

The items above are sorted by increasing difficulty and extra credit.
Do not start working on extra credit if you are not confident your
VCG is working properly for the simple cases.
%%%%%%%%%%%%%%%%%%%%

\end{document}
